\documentclass[12pt]{amsart}
\usepackage[margin=1in]{geometry}
\usepackage[utf8]{inputenc}
\usepackage{amsfonts}
\usepackage{amsthm}
\usepackage{amsmath}
\usepackage{amssymb}
\usepackage{graphicx}
\usepackage{enumitem}
\usepackage{quiver}
\usepackage{mathrsfs}
\usepackage{amssymb}
\usepackage{mathtools}
\usepackage{hyperref}
\usepackage{cleveref}
\hypersetup{
    colorlinks=true,
    linkcolor=blue,
    filecolor=green,      
    urlcolor=cyan,
    citecolor=magenta,
    pdftitle={Perverse Sheaves, Intersection Cohomology, and Mixed Hodge Modules},
    pdfpagemode=FullScreen,
}
\urlstyle{same}


\newcommand{\R}{\mathbb{R}}
\newcommand{\Q}{\mathbb{Q}}
\newcommand{\V}{\mathbb{V}}
\newcommand{\A}{\mathbb{A}}
\newcommand{\sph}{\mathbb{S}}
\newcommand{\C}{\mathbb{C}}
\newcommand{\Z}{\mathbb{Z}}
\newcommand{\End}{\text{End}}
\newcommand{\Aut}{\text{Aut}}
\newcommand{\Hom}{\text{Hom}}
\newcommand{\id}{\text{id}}

\DeclareMathOperator{\im}{\text{im}}
\DeclareMathOperator{\coker}{\text{coker}}
\DeclareMathOperator{\D}{\mathcal{D}}


\newcommand{\F}{\mathscr{F}}
\newcommand{\sH}{\mathscr{H}}
\newcommand{\ol}{\overline}
\newcommand{\ul}{\underline}
\newcommand{\pH}{\prescript{p}{}{\mathcal{H}}}

\newtheorem{theorem}{Theorem}[section]
\newtheorem{proposition}[theorem]{Proposition}
\newtheorem{lemma}[theorem]{Lemma}
\newtheorem{corollary}[theorem]{Corollary}

\theoremstyle{definition}
\newtheorem{definition}[theorem]{Definition}
\newtheorem{example}[theorem]{Example}


\theoremstyle{remark}
\newtheorem{remark}[theorem]{Remark}

\numberwithin{equation}{section}




\title{Perverse Sheaves and Intersection Cohomology}
\author{Edward Varvak}

\begin{document}

\maketitle

These are notes I wrote up for my own comprehension while following the text of
Hotta--Takeuchi--Tanisaki \cite{HTT}. No part of this work is original, save for 
any errors I may have introduced throughout my reading.

\section{T-Structures}
\begin{definition}
    Let $\D$ be a triangulated category, and $\D^{\leq 0}, \D^{\geq 0}$ its full subcategories. Set $\D^{\leq n} = \D^{\leq 0}[n]$. The pair $(\D^{\leq 0}, \D^{\geq 0})$ defines a t-structure on $\D$ if
    \begin{itemize}
        \item $\D^{\leq -1} \subseteq \D^{\leq 0}, \D^{\geq 1} \subseteq \D^{\geq 0}$
        \item For all $X \in \D^{\leq 0}$ and $Y \in \D^{\geq 1}$ we have $\Hom_{\D}(X, Y) = 0.$
        \item For any $X \in \D$ there exists an exact triangle
        \begin{equation}
            X_0 \to X \to X_1 \overset{+1}{\to} \cdots
        \end{equation}
        such that $X_0 \in \D^{\leq 0}$ and $X_1 \in \D^{\geq 1}$.
    \end{itemize}
    The full subcategory $\mathcal{C} = \D^{\geq 0} \cap \D^{\leq 0}$ is called the heart of the t-structure $(\D^{\leq 0}, \D^{\geq 0})$.
\end{definition}

The heart of a t-structure allows us to get a good grasp on the ambient derived category.
\begin{theorem}
    $\mathcal{C} = \D^{\geq 0} \cap \D^{\leq 0}$ is an Abelian category, and any exact sequence
    \begin{equation*}
        0 \to X \to Y \to Z \to 0
    \end{equation*}
    gives rise to a distinguished triangle
    \begin{equation*}
        X \to Y \to Z \overset{+1}{\to} \cdots
    \end{equation*}
    in $\D$.
\end{theorem}

This allows us to define the functor $H^0$, which sends our derived category to its heart.
\begin{definition}
    The functor
    \begin{equation*}
        H^0: \D \to \mathcal{C} = \D^{\geq 0} \cap \D^{\leq 0}, \qquad
        X \mapsto \tau^{\leq 0} \tau^{\geq 0}X.
    \end{equation*}
    Likewise, define $H^n(X) = H^0(X[n])$.
\end{definition}
Now let $F: \D_1 \to \D_2$ be a functor of triangulated categories. Denote by $\mathcal{C}_1$, $\mathcal{C}_2$ the hearts of the respective categories.
\begin{definition}
    The additive functor
    \begin{equation*}
        \prescript{p}{}{F}: \mathcal{C}_1 \to \mathcal{C}_2,\qquad
        X \mapsto H^0(F(X)).
    \end{equation*}
    We say that $\prescript{p}{}{F}$ is left t-exact if $F(\D_1^{\geq 0}) \subseteq \D_2^{\geq 0}$, and analogously for right t-exactness.
\end{definition}

\section{Perverse sheaves}

Now we define a t-structure on the derived category of constructable sheaves. Recall that we have a Verdier duality functor $\mathbf{D}_X: D^b_{con}(X)^{op} \to D^b_{con}(X)$.
\begin{definition}
    For $F^\bullet \in \D^b_{con}(X)$, we have $F^\bullet \in \D^{\leq 0}_{con}(X)$ if and only if 
    \[\dim \text{supp}( H^j(F^\bullet)) \leq -j\] for all $j \in \Z$. 
    Likewise, we have $F^\bullet \in \D^{\geq 0}_{con}(X)$ if and only if 
    \[\dim \text{supp}( H^j(\mathbf{D}_X F^\bullet)) \leq -j\] for all $j \in \Z$.
\end{definition}

It is not obvious that this is indeed a t-structure, so it must be stated seperately.
\begin{theorem}
    The pair $(\D_{con}^{\leq 0}, \D_{con}^{\geq 0})$ forms a t-structure. 
\end{theorem}

\begin{definition}
    We call the heart of this t-structure $\text{Perv}(\C_X)$ the perverse sheaves on $X$. The functor $\mathbf{D}_X$ induces a duality of $\text{Perv}(\C_X)$.
    Now, for any functor $F: \D^b_{con}(X) \to \D^b_{con}(Y)$,
    denote by $\prescript{p}{}{F}$ the composition $\prescript{p}{}{H^0} \circ F$, viewed as a functor $\text{Perv}(\C_X) \to \text{Perv}(\C_Y)$.
    In particular, we get the associated functors
    \begin{equation*}
        \prescript{p}{}{f^{-1}},  \prescript{p}{}{f^{!}}: 
        \text{Perv}(\C_Y) \to \text{Perv}(\C_X),
    \end{equation*}
    and
    \begin{equation*}
        \prescript{p}{}{Rf_*}, \prescript{p}{}{Rf_!}:
        \text{Perv}(\C_X) \to \text{Perv}(\C_Y).
    \end{equation*}
\end{definition}



% Suppose that a space $X$ decomposes as $Z \sqcup U$ for a closed $i: Z \to X$ a proper map and $j: U \to X$ an open embedding. Notice that there always exists a natural map
% \begin{equation}
%     i_! \F \to i_* \F,\quad s \mapsto s,
% \end{equation}
% sending a compactly supported section to itself. Clearly, if $i$ is as above, this is an isomorphism for all $\F$, so that $i_! = i_*$.
% Likewise, we have an equality $j^! = j^*$.
% Importantly, we get an exact triangle
% \begin{equation}
%     i_! i^! \to \text{id} \to j_* j^* \overset{[+1]}{\to} \cdots,\qquad 
%     j_! j^! \to \text{id} \to i_* i^* \overset{[+1]}{\to} \cdots.
% \end{equation}

% Now we can define perverse sheaves in terms of this language. Take $X = \bigsqcup_\lambda X_\lambda$ to be a Whitney stratification.
% Denote $i_\lambda: X_\lambda \to X$ the corresponding inclusion, and $d_\lambda$ the dimension of $X_\lambda$.

% \begin{definition}
%     $\sideset{^p}{^{\leq 0}_\lambda}{\D}:= \{
%         \F \in D^b_{const}(X_\lambda)\ |\ \sH^i(\F) = 0\ \forall i > -d_\lambda
%     \}$,\newline
%     $\sideset{^p}{^{\geq 0}_\lambda}{\D}:= \{
%         \F \in D^b_{const}(X_\lambda)\ |\ \sH^i(\F) = 0\ \forall i < -d_\lambda
%     \}$.
% \end{definition}


% With this in mind, we are ready to consider our first example of a perverse sheaf.
% \begin{example}
% Let $X = \mathbb{P}^1$, with $Z = \{p\}$ a point. Then 

% \end{example}

Next, we define minimal extensions of perverse sheaves. Let $X = Z \sqcup U$ with $i: Z \to X$ proper and $j: U \to X$ open.
For $F^\bullet \in \D^b_{con}(X)$, we have natural maps
\begin{equation*}
    j_! F^\bullet \to Rf_* F^\bullet,\qquad s \mapsto s,
\end{equation*}
mapping a section to itself. If $F^\bullet$ is a perverse sheaf on $U$, we also get maps
\begin{equation} \label{eq-minimal-ext}
    \prescript{p}{}{j_!} F^\bullet \to \prescript{p}{}{f_*} F^\bullet.
\end{equation}

\begin{definition}
    We denote by $\prescript{p}{}{f_{!*}} F^\bullet$ the image of \eqref{eq-minimal-ext}, and call it the minimal extension of $F^\bullet \in \text{Perv}(\C_U)$.
\end{definition}
In other words, we have maps
\begin{equation}
    \prescript{p}{}{j_!} F^\bullet \twoheadrightarrow \prescript{p}{}{j_{!*}} F^\bullet
    \hookrightarrow \prescript{p}{}{j_*} F^\bullet.
\end{equation}

\begin{theorem}
    Let $F^\bullet \in \text{Perv}(\C_U)$ as above. Then
    \begin{itemize}
        \item[(i)] $\prescript{p}{}{f_*} F^\bullet$ has no non-trivial sub-objects whose support is contained in $Z$.
        \item[(ii)] $\prescript{p}{}{j_!} F^\bullet$ has no non-trivial quotient object whose support is contained in $Z$.
        \item[(iii)] $\prescript{p}{}{j_{!*}} F^\bullet$ has neither sub- or quotient objects whose support is contained in $Z$.
    \end{itemize}
\end{theorem}

Now suppose that $F^\bullet$ is a simple object in $\text{Perv}(\C_U)$.
We claim that $\prescript{p}{}{j}_{!*} F^\bullet$ is also simple in 
$\text{Perv}(\C_X)$. Indeed, for any subobject 
$G^\bullet \subseteq \prescript{p}{}{j}_{!*} F^\bullet$ we may consideer the exact sequence
\begin{equation}
    0 \to G^\bullet \to \prescript{p}{}{j}_{!*} F^\bullet \to H^\bullet \to 0.
\end{equation}
Since $j^! = j^{-1}$ is exact here, we may apply it to obtain
\begin{equation}
    0 \to j^{-1} G^\bullet \to F^\bullet \to j^{-1} H^\bullet \to 0.
\end{equation}
Since $F^\bullet$ is simple, $j^{-1}G^\bullet$ or $j^{-1} H^\bullet$ is zero,
or in other words, one of $G^\bullet$ and $H^\bullet$ is supported in $Z$.
By the previous theorem, this makes it zero.

\begin{theorem} \label{minimal-characterization}
    $G = \prescript{p}{}{j_{!*}} F^\bullet$ is the unique perverse sheaf 
    satisfying the conditions
    \begin{itemize}
        \item[(i)] $G^\bullet|_U \simeq F^\bullet$,
        \item[(ii)] $i^{-1} G^\bullet \in 
        \prescript{p}{}{\D}^{\leq -1}_{con}(Z)$,
        \item[(iii)] \(i^! G^\bullet \in 
        \prescript{p}{}{\D}^{\geq 1}_{con}(Z)\).
    \end{itemize}
\end{theorem}




\section{Intersection cohomology}

Let $X$ be an irreducible complex projective variety (or an irreducible compact analytic space) of dimension $d$. We may start by defining the intersection cohomology complex in the following way.
\begin{definition}
    Take $U \subseteq X^{\text{reg}}$ to be a Zariski open subset of the smooth locus of $X$. For $\underline{\C}_U[d]$ the constant perverse sheaf on $U$, define $IC_X^{\bullet}$ to be a minimal extension to all of $X$.
\end{definition}

We may explicitly construct $IC_X^{\bullet}$. Fix a Whitney stratification $X = \bigsqcup_\alpha X_{\alpha}$, 
\begin{equation}
    X_k = \coprod_{\dim X_\alpha \leq k} X_\alpha, \quad U_k := X \smallsetminus X_{k-1},
\end{equation}
with each $X_k \smallsetminus X_{k-1}$ a smooth $k$-dimensional complex manifold. We have a family of maps $j_k: U_k \hookrightarrow U_{k-1}$, for $k=1,\dots, d$. Altogether, this fits into
\begin{equation}
    \emptyset \hookrightarrow U_d \hookrightarrow U_{d-1} \hookrightarrow \cdots
    \hookrightarrow U_1 \hookrightarrow X.
\end{equation}

\begin{theorem} \label{ic-structure}
    \(IC_X^{\bullet}\) is quasi-isomorphic to the complex
    \begin{equation}
        \prescript{p}{}{j_{!*}}(\C_U[d_X]) \simeq 
        \left(\tau^{\leq -1}Rj_{1*}\right) \circ \left(\tau^{\leq -2}Rj_{2*}\right)
        \circ \cdots \circ \left(\tau^{\leq -d}Rj_{d*}\right)
        \left(\underline{\C}_U[d]\right).
    \end{equation}
\end{theorem}
\begin{proof}
Suppose that $F^\bullet$ is a perverse sheaf on $U_k$ whose restriction to 
any strata $X_\alpha \subseteq U_k$ has locally constant $\sH^i$.
We show that $G^\bullet := \tau^{\leq -k} Rj_{k*}(F^\bullet)$ satisfies the
unique characterization of a minimal extension from Theorem 
\Cref{minimal-characterization}. First, note that $U_k$ consists of 
strata of dimension $\geq k$, so that \(\sH^r(F^\bullet)  = 0\) for 
\(r > -k\). In particular,
\begin{equation*}
    \left[\tau^{\leq -k} Rj_{k*} F^\bullet\right]|_{U_k} \simeq F^\bullet,
\end{equation*}
so that what we have is indeed an extension of $F^\bullet$, satisfying (i).
Next, set \(Z := U_{k-1} \smallsetminus U_k = 
\bigsqcup_{\dim X_\alpha = k-1} X_\alpha\). Denote by $i: Z \to U_k$ the
associated closde embedding. Then $i^{-1} G^\bullet$ has locally constant 
cohomology sheaves on each $X_\alpha \subseteq Z$, so that 
\(\sH^r(i^{-1} G^\bullet) = 0\) for \(r > -k\). This implies that 
\(i^{-1} G^\bullet \in \prescript{p}{}{\D}_{con}^{\leq -1},\) thus satisfying
condition (ii). Finally, consider the triangle
\begin{equation}
    G \to Rj_{k*} F^\bullet \to \tau^{\geq -k+1} Rj_{k*} F^\bullet
    \overset{+1}{\to} \cdots
\end{equation}
Applying $i^!$, we note that the middle vanishes. This gives us isomorphisms
\begin{equation*}
    i^! G^\bullet \simeq i^! (\tau^{\geq -k+1} Rj_{k*} F^\bullet)[-1]
\end{equation*}
so that $\sH^r(i^! G^\bullet) = 0$ for \(r < -k\). But since \(i^! G\) has locally
constant cohomology on \(X_\alpha \subseteq Z\), we get
\(i^! G \in \prescript{p}{}{\D}_{con}^{\geq 1}(Z)\), proving (iii).
\end{proof}

Importantly, the intersection cohomology complex is self-dual in the way we expect of perverse sheaves.

\begin{theorem}\label{canonical-morphisms}
    \(IC_X^{\bullet} \simeq \mathbb{D}_X (IC_X^{\bullet}) \). Furthermore, there exist canonical morphisms
    \begin{equation}
        \underline{\C}_X \to IC_X^{\bullet}[-d] \to \omega_X^\bullet[-2d].
    \end{equation}
\end{theorem}
\begin{proof}
We have an isomorphism in the derived category of constructible sheaves
from above
\begin{equation}
    \tau^{\leq -d}\ \prescript{p}{}{j}_{!*} \underline{\C}_X[d] \simeq
     Rj_{1*} \circ \cdots \circ Rj_{d*}
     \left(\underline{\C}_X[d]\right) \simeq (j_* \C)[d].
\end{equation}
This is none other than the intersection complex of $X$, and we see that it admits a canonical map from $\underline{\C}_X$. The second map in the composition comes from taking the Verdier dual,
\(IC_X^\bullet[d] \to \omega_X^\bullet \).
\end{proof}

With the intersection complex in place, we are ready to define intersection cohomology.
\begin{definition}
    For $i \in \Z$, we define
    \begin{equation}
        IH^i(X) = H^i\left(R\Gamma\left(X, IC_X^\bullet[-d]\right)\right),
    \end{equation}
    \begin{equation}
        IH_c^i(X) = H^i\left(R\Gamma_c\left(X, IC_X^\bullet[-d]\right)\right).
    \end{equation}
\end{definition}

Importantly, the intersection cohomology of $X$ satisfies Poincare duality.
\begin{theorem}
    Let $X$ be irreducible of dimension $d$. Then
    \begin{equation}
        IH^i(X) \simeq \left[IH_c^{2d-i}(X)\right]^*.
    \end{equation}
\end{theorem}
\begin{proof}
Let $a_X: X \to \{p\}$ be the unique map to a point. We have 
\begin{equation}
    R\Hom_\C(Ra_{X!} IC^\bullet, \C) \simeq 
    Ra_{X*} R\Hom_{\C_X}(IC_X^\bullet, \omega_X^\bullet).
\end{equation}
by Verdier duality. But \(IC_X^\bullet\) is self-dual, so that
\begin{equation*}
    R\Hom_{\C_X}(IC_X^\bullet, \omega_X^\bullet) = 
    \mathbf{D}_X(IC_X^\bullet) = IC_X^\bullet.
\end{equation*}
Thus we get an isomorphism
\begin{equation}
    \left[R \Gamma_c(X, IC_X^\bullet)\right]^* \simeq 
    R\Gamma(X, IC_X^\bullet).
\end{equation}


\end{proof}

\begin{example}
Let $X$ be an irreducible complex projective variety with isolated singular points $p_1,\dots, p_k$. Then it suffices to consider the stratification
\[ X = \{p_1,\dots,p_k\} \sqcup X^{\text{reg}}. \]
Then we have \(X_0 = \dots = X_{d-1} = \{p_1,\dots, p_k\}\), and \(X_d = X\). Thus, we get the inclusions of complex manifolds
\[ \emptyset \to U_d = X^{\text{reg}} = U_1 \overset{j_1}{\to} X.\]
The only interesting map in this stratification is $j_1$, the inclusion of the smooth locus. In particular, by Theorem \ref{ic-structure},
\[ IC_X^\bullet \simeq \tau^{\leq -1}\left(Rj_{1*} \underline{\C}_{U_1}\right). \] 
Now, associated to truncation is the exact couple
\begin{equation}
    IC_X^\bullet[-d] \to Rj_{1*} \underline{\C}_{U_1} \to 
    \tau^{\geq d}\left(Rj_{1*} \underline{\C}_{U_1}\right) \overset{+1}{\to} \cdots
\end{equation}
But now we can apply $R\Gamma$ and take cohomology. For $0 \leq i < d$, we have
\( IH^i(X) = H^i(X, Rj_{1*}\overline{\C}_{U_1}) = H^i(X^{\text{reg}}, \C). \) For $i=d$, we do not get a clear vanishing on the right, but the map
\(IH^d(X) \to H^d(X^{\text{reg}}, \C)\) is injective. To analyze this case more carefully, consider the canonical morphism
\( \underline{\C}_X \to IC_X^\bullet[-d] \)
guarenteed by Theorem \Cref{canonical-morphisms}.

We may associate to this natural map a new exact couple
\begin{equation}
    \underline{\C}_X \to IC_X^\bullet[-d] \to F^\bullet \overset{+1}{\to} \cdots
\end{equation}
Here \(F^\bullet\) is a constructable sheaf supported only on the zero-dimensional closed subset $\{p_1, \dots, p_k\} = X_0$. In particular, \(H^i(F^\bullet) = 0\) for all $i \geq d$. Thus, we get \(IH^i(X) = H^i(X, \C)\) for $d < i \leq 2d$. After $i = 2d$, the cohomology of $\underline{\C}_X$ vanishes, and with it $IH^i(X)$. Summarizing,
\begin{equation}
    IH^i(X) = \begin{cases}
        H^i(X^{\text{reg}}, \C) & : 0 \leq i < d \\
        \text{im}(H^i(X, \C) \to H^i(X^{\text{reg}}, \C)) & : i = d \\
        H^i(X, \C) & : d < i \leq 2d \\
        0 & : \text{otherwise}
    \end{cases}
\end{equation}
\qed
\end{example}


\section{Computations via Saito's mixed Hodge modules}
To compute intersection complexes of standard perverse sheaves, we use Saito's 
mixed Hodge modules (\cite{saito88-HM}, \cite{saito90-MHM}). For a quick summary of 
this theory, refer to \cite[\S 2.1]{deC-Rap-Sac21}. Throughout these computations,
we loosely follow \cite[\S 2]{deC-Mig09}.

Let \(X\) be a variety over \(\C\). Saito \cite{saito90-MHM} defines a category
\(\D^b \mathrm{MHM}(X)\) of complexes of mixed Hodge modules, equipped with
Hodge-theoretic weight formalism and a natural morphism
\(\mathrm{rat}: \D^b \mathrm{MHM}(X) \to \D^b_{con}(X, \Q)\). This category 
has two t-structures: the one corresponding to the standard t-structure on 
\(\D^b_{con}(X, \Q)\), and the perverse t-structure. This sets a heart:
if \(K \in \mathrm{MHM}(X)\), then \(\mathrm{rat}(K) \in \mathrm{Perv}(X, \Q)\).
When restricted to a suitable dense open set, the objects in \(\mathrm{MHM}(X)\) 
become admissable variations of polarizable mixed Hodge structures.

More importantly, for \(f: X \to B\) a proper morphism of varieties and 
\(K \in MHM(X)\), if the derived pushforwards (with respect to the standard t-structures)
\(R^i f_* K\), \(R^i f_! K\) are perverse sheaves, they may be endowed with the 
structure of an object in \(\D^b \mathrm{MHM}(X)\). More concretely, 
we have the following theorem.

\begin{theorem}[Decomposition Theorem]
    Let \(K \in \mathrm{MHM}(X)\) be of pure weight \(w\). Then there is a splitting 
    of \(Rf_* K\) with respect to the perverse t-structure 
    \begin{equation}
        Rf_* K = \bigoplus_{k} \pH^k(Rf_* K)[-k],
    \end{equation}
    where each summand \(\pH^k(Rf_* K)[-k]\) is pure and semisimple in 
    \(\D^b \mathrm{MHM}(X)\) of weight \(w-k\). (Should there be no \([-k]\) shift?) 
    Moreover, if \(\eta\) is an ample class on \(X\), there is a relative Hard 
    Lefschetz isomorphism
    \begin{equation}
        \eta^i: \pH^{-i}(Rf_* K) \simeq \pH^i(Rf_* K).
    \end{equation}
\end{theorem}

\subsection{Fibrations of smooth algebraic varieties.}
In this section, we focus on the case of \(f: X \to B\), where \(f\) is proper with 
positive-dimensional fibers, and \(X\) and \(B\) are smooth. 

\begin{example}
Let \(X\) be a projective surface and \(B\) a curve. Denote by \(B^\circ\) the locus 
of regular values of \(f\), \(\Delta = B \smallsetminus B^\circ\). Then
\(R^1 f_* \Q_X|_{B^\circ}\) is a weight one variation of Hodge structures. Denote 
the underlying local system by \(\mathbb{V}\).

In order for \(\mathbb{V}\) to appear in middle perversity in the decomposition theorem, 
we must shift the pushforward by \(2\): set \(W = Rf_* \Q_X[2]\).
\begin{align*}
    W|_{B^\circ} &= \pH^{-2}(W|_{B^\circ})[2] \oplus 
    \pH^{-1}(W|_{B^\circ})[1] \oplus \pH^0(W|_{B^\circ})[0] \\
    &= \Q_{B^\circ}[2] \oplus \mathbb{V}[1] \oplus \Q_{B^\circ}.
\end{align*}

Next, we set our attention to the singular fibers of the pushforward. Let 
\(p \in \Delta\) be a critical value. Then
\begin{align*}
    \pH^{-2}(W)_p[2] &= H^0(X_p, \Q)_p[2] = \Q_p[2], \\
    \pH^{-1}(W)_p[1] &= H^1(X_p, \Q)_p[1], \\
    \pH^{-2}(W)_p[0] &= H^2(X_p, \Q)_p[0] = \bigoplus_{X_p^{(i)} \subseteq X_p}
    \left<[X_p^{(i)}]\right>_p[0],
\end{align*}
where the last sum is taken over the irreducible components of \(X_p\).

Altogether, the decomposition theorem yields
\begin{equation} \label{eq-example-decomp-wt1-loc}
    W = \Q_B[2] \oplus \ol{\mathbb{V}}[1] \oplus \Q_B[0] \oplus \bigoplus_{p \in \Delta} 
    \left(\bigoplus_{X_p^{(i)} \subseteq X_p} 
    \frac{\left<[X_p^{(i)}]\right>}{\left<[X_p]\right>}\right)_p[0].
\end{equation}
Here \(\ol{\mathbb{V}} = R^1 f_* \Q_X\). We may also compute the intersection complex 
\(IC_B(\mathbb{V})\) valued in the local system \(\mathbb{V}\). It appears in this 
decomposition, and indeed, it is clear from the perverse degrees that 
\begin{equation}
    IC_B(\mathbb{V}) = \ol{\mathbb{V}}[1] \oplus \bigoplus_{p \in \Delta} 
    \left(\bigoplus_{X_p^{(i)} \subseteq X_p} 
    \frac{\left<[X_p^{(i)}]\right>_p}{\left<[X_p]\right>_p}\right)[0].
\end{equation}
This is identified with \(R^1 f_* \Q_X[1]\) if and only if the fibers of \(f\) are 
all irreducible.
\qed
\end{example}


\begin{example}
We now consider the same example, but on the level of mixed Hodge modules. As a Hodge
module, \(\Q_X(0)\) is pure of weight 0. Over \(B^\circ\), its pushforward 
decomposes as 
\begin{equation}
    \Q_{B^\circ}(0)[0] \oplus \mathcal{V}[-1] \oplus \Q_{B^\circ}(-1)[-2],
\end{equation}
where \(\mathcal{V}\) is the weight one variation of Hodge structures associated to the 
local system \(\mathbb{V}\) above. Over \(p \in \Delta\), we get 
\begin{equation}
    Rf_* \Q_X|_p = H^0(X_p, \Q)[0] \oplus H^1(X_p, \Q)[-1] \oplus H^2(X_p, \Q)[-2].
\end{equation}
Hence, we get the same decomposition as in \eqref{eq-example-decomp-wt1-loc}: 
\begin{equation}
   Rf_* \Q_X = \Q_B(0)[0] \oplus \ol{\mathcal{V}}[-1] \oplus \Q_B(-1)[-2] \oplus
   \bigoplus_{p \in \Delta} \left(\frac{\left<[X_p^{(1)}], \dots, [X_p^{(n)}]\right>}
   {\left<[X_p^{(1)} + \cdots + X_p^{(n)}]\right>}\right)_p[-2].
\end{equation}
Notice that the relative Hard Lefschetz theorem sends \(\Q_B(0)\) to \(\Q_B(-1)[-2]\)
isomorphically, and fixes all other primative cohomology. We get a mixed Hodge module 
associated to the intersection complex, 
\begin{equation}
    \mathscr{IC}_B(\mathcal{V}) = \ol{\mathcal{V}}[-1] \oplus 
    \bigoplus_{p \in \Delta} \left(\frac{\left<[X_p^{(1)}], \dots, [X_p^{(n)}]\right>}
   {\left<[X_p^{(1)} + \cdots + X_p^{(n)}]\right>}\right)_p[-2].
\end{equation}
Here \(\ol{\mathcal{V}} = R^1 f_* \Q_X\). From this description, we see that it is a
simple weight one mixed Hodge module if and only if the fibers of \(f\) are irreducible.

We may compute \(IH^k(B, \mathcal{V}) = H^k(R \Gamma(\mathscr{IC}_B(\mathcal{V})))\).
By Saito, these carry mixed Hodge structures. In this case, we have 
\(IH^1(B, \mathcal{V}) = H^0(\ol{\mathcal{V}})\) the non-vanishing invariant cycles, and
\(IH^2(B, \mathcal{V}) = \bigoplus_{p \in \Delta} H^2(X_p, \Q)^{\mathrm{prim}}\)
the primative weight two cohomogy of the singular fibers.
\qed
\end{example}



\bibliographystyle{alpha}
\bibliography{refs}


\end{document}
